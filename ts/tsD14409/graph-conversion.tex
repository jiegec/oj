% Created 2018-03-11 Sun 13:16
% Intended LaTeX compiler: pdflatex
\documentclass[11pt]{article}
\usepackage[utf8]{inputenc}
\usepackage[T1]{fontenc}
\usepackage{graphicx}
\usepackage{grffile}
\usepackage{longtable}
\usepackage{wrapfig}
\usepackage{rotating}
\usepackage[normalem]{ulem}
\usepackage{amsmath}
\usepackage{textcomp}
\usepackage{amssymb}
\usepackage{capt-of}
\usepackage{hyperref}
\usepackage[newfloat]{minted}
\usepackage{xeCJK}
\setCJKmainfont{Songti SC}
\usepackage{latexsym}
\usepackage[mathscr]{eucal}
\usepackage[section]{placeins}
\usepackage{float}
\usepackage{svg}
\author{计72 陈嘉杰}
\date{\today}
\title{不同的图的格式之间的转换 tsD14409}
\hypersetup{
 pdfauthor={计72 陈嘉杰},
 pdftitle={不同的图的格式之间的转换 tsD14409},
 pdfkeywords={},
 pdfsubject={},
 pdfcreator={Emacs 26.0.91 (Org mode 9.1.7)},
 pdflang={English}}
\begin{document}

\maketitle
\tableofcontents

\section{题目说明}
\label{sec:orgb461eab}
文件 \texttt{input.txt} 中含有一个图的权矩阵表示,要求输出这个图的关联矩阵、边列表、正向表和邻接表到 \texttt{output.txt} 中。

\section{实现思路}
\label{sec:orgc182ded}
先把传入的这个权矩阵转化为边的列表和邻接表。边列表用于关联矩阵和边列表的转换,邻接表可以用于正向表的生成。

\section{实现步骤}
\label{sec:org9a66093}
\subsection{下标约定}
\label{sec:orgf8a23b4}
输入数据中点和边都从1开始,故在我的代码中同样如此。部分与点和边无关的下标,则从0开始。
\subsection{数据的读入}
\label{sec:org922e68e}

考虑到输入的数据中 0 < n \(\le\) 100, 所以可以直接在全局变量中开足够大的空间存放数据。
首先是文件重定向,接着,读入数据,转换成边的列表和邻接表的格式。

\begin{minted}[linenos,numbersep=5pt,breaklines]{c++}
#include <stdio.h>

int edge_from[10010];
int edge_to[10010];
int edge_weight[10010];
int edge_num = 1;
int adj[110][10010] = {{0}};
int adj_weight[110][10010] = {{0}};
int adj_num[110] = {0};

int main() {
  freopen("input.txt", "r", stdin);
  freopen("output.txt", "w", stdout);
  scanf("%d", &n);
  for (int i = 1; i <= n; i++) {
    for (int j = 1; j <= n; j++) {
      int weight;
      scanf("%d", &weight);
      if (weight) {
        edge_from[edge_num] = i;
        edge_to[edge_num] = j;
        edge_weight[edge_num] = weight;
        edge_num++;

        adj[i][adj_num[i]] = j;
        adj_weight[i][adj_num[i]] = weight;
        adj_num[i] ++;
      }
    }
  }
  return 0;
}
\end{minted}

\subsection{输出关联矩阵}
\label{sec:org00c5648}
关联矩阵中每一列对应一条边,只要把这条边的两个结点所在的行进行处理即可。处理完则输出。

\begin{minted}[linenos,numbersep=5pt,breaklines]{c++}
// guanlian
for (int i = 1; i < edge_num; i++) {
  link_matrix[edge_from[i]][i] = 1;
  link_matrix[edge_to[i]][i] = -1;
 }
for (int i = 1; i <= n; i++) {
  for (int j = 1; j < edge_num; j++) {
    printf("%d ", link_matrix[i][j]);
  }
  printf("\n");
 }
\end{minted}

\subsection{输出边列表}
\label{sec:orgba5faeb}
边列表,在读入数据的时候已经处理好,简单处理成矩阵的形式,直接输出即可。

\begin{minted}[linenos,numbersep=5pt,breaklines]{c++}
// bianliebiao
for (int i = 1; i < edge_num; i++) {
  edge_list[0][i] = edge_from[i];
  edge_list[1][i] = edge_to[i];
  edge_list[2][i] = edge_weight[i];
 }

for (int i = 0; i < 3; i++) {
  for (int j = 1; j < edge_num; j++) {
    printf("%d ", edge_list[i][j]);
  }
  printf("\n");
 }
\end{minted}

\subsection{输出正向表}
\label{sec:orgc825de7}
这个可能就是最难得一部分了。这个表中,A数组的元素代表相应的结点的后继结点的起始下标。最后添加一个结尾。所以,我们使用变量 \texttt{zhengxiangbiao\_current} 记录当前的下标,以此更新A数组。

\begin{minted}[linenos,numbersep=5pt,breaklines]{c++}
  // zhengxiangbiao
for (int i = 1; i <= n; i++) {
  zhengxiangbiao_a[i] = zhengxiangbiao_current;
  for (int j = 0; j < adj_num[i]; j++) {
    zhengxiangbiao_b[zhengxiangbiao_current] = adj[i][j];
    zhengxiangbiao_z[zhengxiangbiao_current] = adj_weight[i][j];
    zhengxiangbiao_current ++;
  }
 }
zhengxiangbiao_a[n+1] = zhengxiangbiao_current;

for (int i = 1; i <= n+1; i++) {
  printf("%d ", zhengxiangbiao_a[i]);
 }
printf("\n");

for (int i = 1; i < zhengxiangbiao_current; i++) {
  printf("%d ", zhengxiangbiao_b[i]);
 }
printf("\n");

for (int i = 1; i < zhengxiangbiao_current; i++) {
  printf("%d ", zhengxiangbiao_z[i]);
 }
printf("\n");
\end{minted}

\subsection{输出邻接表}
\label{sec:orgc4b3c87}
这一步也很简单,在读入数据的时候已经处理完毕。

\begin{minted}[linenos,numbersep=5pt,breaklines]{c++}
// linjiebiao
for (int i = 1;i <= n;i++) {
  for (int j = 0;j < adj_num[i];j++) {
    printf("%d %d ", adj_weight[i][j], adj[i][j]);
  }
  printf("\n");
 }
\end{minted}

\subsection{完整代码}
\label{sec:org6ca54fa}
\begin{minted}[linenos,numbersep=5pt,breaklines]{c++}
#include <stdio.h>

int edge_from[10010];
int edge_to[10010];
int edge_weight[10010];
int edge_num = 1;
int adj[110][10010] = {{0}};
int adj_weight[110][10010] = {{0}};
int adj_num[110] = {0};
int link_matrix[110][10010] = {{0}};
int edge_list[3][10010] = {0};
int zhengxiangbiao_a[110] = {0};
int zhengxiangbiao_b[10010] = {0};
int zhengxiangbiao_z[10010] = {0};
int zhengxiangbiao_current = 1;
int n;

int main() {
  freopen("input.txt", "r", stdin);
  freopen("output.txt", "w", stdout);
  scanf("%d", &n);
  for (int i = 1; i <= n; i++) {
    for (int j = 1; j <= n; j++) {
      int weight;
      scanf("%d", &weight);
      if (weight) {
        edge_from[edge_num] = i;
        edge_to[edge_num] = j;
        edge_weight[edge_num] = weight;
        edge_num++;

        adj[i][adj_num[i]] = j;
        adj_weight[i][adj_num[i]] = weight;
        adj_num[i] ++;
      }
    }
  }

  // guanlian
  for (int i = 1; i < edge_num; i++) {
    link_matrix[edge_from[i]][i] = 1;
    link_matrix[edge_to[i]][i] = -1;
  }
  for (int i = 1; i <= n; i++) {
    for (int j = 1; j < edge_num; j++) {
      printf("%d ", link_matrix[i][j]);
    }
    printf("\n");
  }

  // bianliebiao
  for (int i = 1; i < edge_num; i++) {
    edge_list[0][i] = edge_from[i];
    edge_list[1][i] = edge_to[i];
    edge_list[2][i] = edge_weight[i];
  }

  for (int i = 0; i < 3; i++) {
    for (int j = 1; j < edge_num; j++) {
      printf("%d ", edge_list[i][j]);
    }
    printf("\n");
  }

  // zhengxiangbiao
  for (int i = 1; i <= n; i++) {
    zhengxiangbiao_a[i] = zhengxiangbiao_current;
    for (int j = 0; j < adj_num[i]; j++) {
      zhengxiangbiao_b[zhengxiangbiao_current] = adj[i][j];
      zhengxiangbiao_z[zhengxiangbiao_current] = adj_weight[i][j];
      zhengxiangbiao_current ++;
    }
  }
  zhengxiangbiao_a[n+1] = zhengxiangbiao_current;

  for (int i = 1; i <= n+1; i++) {
    printf("%d ", zhengxiangbiao_a[i]);
  }
  printf("\n");

  for (int i = 1; i < zhengxiangbiao_current; i++) {
    printf("%d ", zhengxiangbiao_b[i]);
  }
  printf("\n");

  for (int i = 1; i < zhengxiangbiao_current; i++) {
    printf("%d ", zhengxiangbiao_z[i]);
  }
  printf("\n");

  // linjiebiao
  for (int i = 1;i <= n;i++) {
    for (int j = 0;j < adj_num[i];j++) {
      printf("%d %d ", adj_weight[i][j], adj[i][j]);
    }
    printf("\n");
  }
  return 0;
}

\end{minted}

\section{遇到的问题和得到的收获}
\label{sec:org4734df3}
遇到的问题主要就是审题不仔细,忘记输出权值,因此有了两次的错误提交。改掉之后就成功 AC 了。得到的收获就是,即便是写一个很简单的程序,最好也要提前想好思路和容易写错的地方随时提醒自己。这样可以省出更多的时间,省出时间来完成其它的作业。
\end{document}
