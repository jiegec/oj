% Created 2018-03-24 Sat 19:08
% Intended LaTeX compiler: pdflatex
\documentclass[11pt]{article}
\usepackage[utf8]{inputenc}
\usepackage[T1]{fontenc}
\usepackage{graphicx}
\usepackage{grffile}
\usepackage{longtable}
\usepackage{wrapfig}
\usepackage{rotating}
\usepackage[normalem]{ulem}
\usepackage{amsmath}
\usepackage{textcomp}
\usepackage{amssymb}
\usepackage{capt-of}
\usepackage{hyperref}
\usepackage[newfloat]{minted}
\usepackage{xeCJK}
\setCJKmainfont{Songti SC}
\usepackage{latexsym}
\usepackage[mathscr]{eucal}
\usepackage[section]{placeins}
\usepackage{float}
\usepackage{svg}
\author{计72 陈嘉杰}
\date{\today}
\title{Dijstra 算法 tsD14536}
\hypersetup{
 pdfauthor={计72 陈嘉杰},
 pdftitle={Dijstra 算法 tsD14536},
 pdfkeywords={},
 pdfsubject={},
 pdfcreator={Emacs 26.0.91 (Org mode 9.1.7)},
 pdflang={English}}
\begin{document}

\maketitle
\tableofcontents

\section{题目说明}
\label{sec:org312c95a}
文件 \texttt{input.txt} 中含有一个图的权矩阵表示,要求输出单源最短路径到 \texttt{output.txt} 中。

\section{实现思路}
\label{sec:org791f21b}
实现上课在PPT中描述的O(n\(^{\text{2}}\))的Dijkstra算法,并没有编写带优先队列优化的版本。

\section{程序编译环境}
\label{sec:org13b00ef}
\begin{enumerate}
\item 操作系统: macOS
\item 编译器: LLVM/Clang 6.0.0
\end{enumerate}

\section{实现步骤}
\label{sec:org96b6a63}
\subsection{下标约定}
\label{sec:orgffcde3e}
输入数据中点和边都从1开始,故在我的代码中同样如此。

\subsection{数据的读入}
\label{sec:orgc08588e}

考虑到输入的数据中 0 < n < 32, 所以可以直接在全局变量中开足够大的空间存放数据。
首先是文件重定向,接着,读入数据,保存权矩阵,并初始化距离向量和已访问的点集。

\begin{minted}[linenos,numbersep=5pt,breaklines]{c++}
#include <stdio.h>

int map[33][33];
int dist[33];
int vis[33];

int main() {
  int n, k;
  freopen("input.txt", "r", stdin);
  freopen("output.txt", "w", stdout);
  scanf("%d", &n);
  for (int i = 1;i <= n;i++) {
    for (int j = 1;j <= n;j++) {
      scanf("%d", &map[i][j]);
    }
    dist[i] = -1;
    vis[i] = 0;
  }
  scanf("%d", &k);
  return 0;
}
\end{minted}

\subsection{初始化源点和源点连出的边}
\label{sec:org3d4ec08}
初始化源点的距离和已访问状态,并根据边初始化后继结点的距离。

\begin{minted}[linenos,numbersep=5pt,breaklines]{c++}
vis[k] = 1;
dist[k] = 0;
for (int i = 1;i <= n;i++) {
  if (map[k][i] != 0) {
    dist[i] = map[k][i];
  }
 }
\end{minted}

\subsection{进行 Dijkstra 算法中的迭代}
\label{sec:org694c152}
每次从未访问的点中找到距离最小的,更新距离向量。

\begin{minted}[linenos,numbersep=5pt,breaklines]{c++}
for (int i = 1;i <= n;i++) {
  int min_dist = 2147483647;
  int min_index = 0;
  for (int j = 1;j <= n;j++) {
    if (vis[j] == 0 && dist[j] != -1 && dist[j] < min_dist) {
      min_dist = dist[j];
      min_index = j;
    }
  }
  if (min_index == 0) {
    break;
  }
  vis[min_index] = 1;
  for (int k = 1;k <= n;k++) {
    if (map[min_index][k] != 0) {
      if (dist[k] == -1 || dist[k] > min_dist + map[min_index][k]) {
        dist[k] = min_dist + map[min_index][k];
      }
    }
  }
 }
\end{minted}

\subsection{输出最短路径的长度}
\label{sec:org80e8b5f}
注意要跳过源点本身。

\begin{minted}[linenos,numbersep=5pt,breaklines]{c++}
  for (int i = 1;i <= n;i++) {
    if (i != k) {
      printf("%d ", dist[i]);
    }
  }
printf("\n");
\end{minted}

\subsection{完整代码}
\label{sec:orgac3b57e}
\begin{minted}[linenos,numbersep=5pt,breaklines]{c++}
#include <stdio.h>

int map[33][33];
int dist[33];
int vis[33];

int main() {
  int n, k;
  freopen("input.txt", "r", stdin);
  freopen("output.txt", "w", stdout);
  scanf("%d", &n);
  for (int i = 1;i <= n;i++) {
    for (int j = 1;j <= n;j++) {
      scanf("%d", &map[i][j]);
    }
    dist[i] = -1;
    vis[i] = 0;
  }
  scanf("%d", &k);
  vis[k] = 1;
  dist[k] = 0;
  for (int i = 1;i <= n;i++) {
    if (map[k][i] != 0) {
      dist[i] = map[k][i];
    }
  }
  for (int i = 1;i <= n;i++) {
    int min_dist = 2147483647;
    int min_index = 0;
    for (int j = 1;j <= n;j++) {
      if (vis[j] == 0 && dist[j] != -1 && dist[j] < min_dist) {
        min_dist = dist[j];
        min_index = j;
      }
    }
    if (min_index == 0) {
      break;
    }
    vis[min_index] = 1;
    for (int k = 1;k <= n;k++) {
      if (map[min_index][k] != 0) {
        if (dist[k] == -1 || dist[k] > min_dist + map[min_index][k]) {
          dist[k] = min_dist + map[min_index][k];
        }
      }
    }
  }
  for (int i = 1;i <= n;i++) {
    if (i != k) {
      printf("%d ", dist[i]);
    }
  }
  printf("\n");
  return 0;
}
\end{minted}

\section{遇到的问题和得到的收获}
\label{sec:orgcdeaeb7}
遇到的问题就是,太长时间没有自己写 Dijkstra ,在编写的时候未能一气呵成,漏写了若干的语句和判断条件。不过,在本地调试以后,很快发现了问题。收获就是,虽然我对这个算法的过程很熟悉了,但是一些细节还是不够清楚,写之前还是需要仔细阅读算法的伪代码,这样可以节省时间。
\end{document}
