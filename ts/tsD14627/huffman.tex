% Created 2018-04-12 Thu 17:08
% Intended LaTeX compiler: pdflatex
\documentclass[11pt]{article}
\usepackage[utf8]{inputenc}
\usepackage[T1]{fontenc}
\usepackage{graphicx}
\usepackage{grffile}
\usepackage{longtable}
\usepackage{wrapfig}
\usepackage{rotating}
\usepackage[normalem]{ulem}
\usepackage{amsmath}
\usepackage{textcomp}
\usepackage{amssymb}
\usepackage{capt-of}
\usepackage{hyperref}
\usepackage[newfloat]{minted}
\usepackage{xeCJK}
\setCJKmainfont{Songti SC}
\usepackage{latexsym}
\usepackage[mathscr]{eucal}
\usepackage[section]{placeins}
\usepackage{float}
\usepackage{svg}
\author{计72 陈嘉杰}
\date{\today}
\title{Huffman 编码 tsD14627}
\hypersetup{
 pdfauthor={计72 陈嘉杰},
 pdftitle={Huffman 编码 tsD14627},
 pdfkeywords={},
 pdfsubject={},
 pdfcreator={Emacs 26.1 (Org mode 9.1.9)},
 pdflang={English}}
\begin{document}

\maketitle
\tableofcontents

\section{题目说明}
\label{sec:org90d842e}
文件 \texttt{input.txt} 中含有一个字符串,然后要把 Huffman 编码后文本的长度写入 \texttt{output.txt} 中。

\section{实现思路}
\label{sec:org282ecfd}
先统计词频,然后构造出哈夫曼树。由于这里只需要输出最后文本的长度,所以只需要对这个树进行遍历,对每个叶子结点进行统计即可得到答案。

\section{程序编译环境}
\label{sec:org3f67926}
\begin{enumerate}
\item 操作系统: macOS
\item 编译器: LLVM/Clang 6.0.0
\end{enumerate}

\section{实现步骤}
\label{sec:orgcc11c91}
\subsection{数据读入}
\label{sec:org74b1706}
\begin{minted}[linenos,numbersep=5pt,breaklines]{c++}
freopen("input.txt", "r", stdin);
freopen("output.txt", "w", stdout);
scanf("%s", buffer);
len = strlen(buffer);
if (len == 1) {
  // only one char?
  // zero entropy here
  // no reasonable answer
  printf("1\n");
  return 0;
 }
\end{minted}

首先把文本都保存到一个足够大的数组中。然后这里有一个有争议的地方:题目数据中有一个点, \texttt{input.txt} 中仅有一个字母 \texttt{a} 。在这种情况下, Huffman 树仅有一个叶结点。此时信息熵为 0 。我不是很确定这里应该写什么答案。还有就是,如果输入的文本是 \texttt{aaaa} ,此时信息熵仍然为 0 。那此时又如何编码呢。

\subsection{建 Huffman 树}
\label{sec:orgd210d60}
首先是数据结构:

\begin{minted}[linenos,numbersep=5pt,breaklines]{c++}
struct Node {
  Node *left = NULL;
  Node *right = NULL;
  int weight = 0;
  int index = 0;

  bool operator()(const Node *a, const Node *b) {
    return a->weight > b->weight;
  }
};
\end{minted}

很直观的一个二叉树, \texttt{weight} 保存子树的词频, \texttt{index} 表示叶结点对应的字母,下面的 \texttt{operator ()} 是提供给 \texttt{priority\_queue} 的比较器。

接下来,统计词频,建树:

\begin{minted}[linenos,numbersep=5pt,breaklines]{c++}
for (int i = 0; i < len; i++) {
  freq[buffer[i] - 'a']++;
 }
for (int i = 0; i < 26; i++) {
  if (freq[i]) {
    Node *node = new Node;
    node->weight = freq[i];
    node->index = i;
    pq.push(node);
  }
 }
while (pq.size() > 1) {
  Node *first = pq.top();
  pq.pop();

  Node *second = pq.top();
  pq.pop();

  Node *new_node = new Node;
  new_node->left = first;
  new_node->right = second;
  new_node->weight = first->weight + second->weight;
  pq.push(new_node);
}
\end{minted}

这就是很常规的建立 Huffman 树的方法。

\subsection{统计结果}
\label{sec:orgc5386b4}
接下来,遍历这棵树,然后统计结果:

\begin{minted}[linenos,numbersep=5pt,breaklines]{c++}
int calc_length(Node *current, int cur_depth) {
  if (current->left == NULL && current->right == NULL) {
    return freq[current->index] * cur_depth;
  }
  return calc_length(current->left, cur_depth + 1) +
         calc_length(current->right, cur_depth + 1);
}

int main() {
  // ..snip..
  root = pq.top();
  printf("%d\n", calc_length(root, 0));
  return 0;
}
\end{minted}

对于字母表中的每个字母,它的高度就是 Huffman 编码中的长度,乘以词频求和即是答案。最后输出即可。

\section{遇到的问题和得到的收获}
\label{sec:org690e3f8}
遇到的问题是,第一, \texttt{priority\_queue} 的默认比较,由于我这里用的是 \texttt{Node *} 类型,所以会变成指针比较。所以需要自己写一个 \texttt{comparator} ,但是全局的 \texttt{operator <} 又不支持指针。所以复用了 \texttt{struct Node} 作为比较器,这样就可以了。第二就是,输入的文本只有一个字母,这个我觉得没有一个很好的答案。为了过 OJ ,只写了一个小小的判断,留下我的疑问。
\end{document}
