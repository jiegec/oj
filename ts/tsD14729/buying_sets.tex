% Created 2018-05-11 Fri 07:27
% Intended LaTeX compiler: pdflatex
\documentclass[11pt]{article}
\usepackage[utf8]{inputenc}
\usepackage[T1]{fontenc}
\usepackage{graphicx}
\usepackage{grffile}
\usepackage{longtable}
\usepackage{wrapfig}
\usepackage{rotating}
\usepackage[normalem]{ulem}
\usepackage{amsmath}
\usepackage{textcomp}
\usepackage{amssymb}
\usepackage{capt-of}
\usepackage{hyperref}
\usepackage[newfloat]{minted}
\usepackage{xeCJK}
\setCJKmainfont{Songti SC}
\usepackage{latexsym}
\usepackage[mathscr]{eucal}
\usepackage[section]{placeins}
\usepackage{float}
\usepackage{svg}
\author{计72 陈嘉杰}
\date{\today}
\title{Buying Sets tsD14729}
\hypersetup{
 pdfauthor={计72 陈嘉杰},
 pdftitle={Buying Sets tsD14729},
 pdfkeywords={},
 pdfsubject={},
 pdfcreator={Emacs 26.1 (Org mode 9.1.9)},
 pdflang={English}}
\begin{document}

\maketitle
\tableofcontents

\section{题目说明}
\label{sec:org15f337c}
输入 n 个集合,每个集合中有一定个数的元素,元素都为 1 到 n 之间的整数。每个集合都有一定的权值,现在需要选取一组集合,使得这组集合元素的并的元素的个数等于选取的集合的个数,并且权值和最小。

\section{实现思路}
\label{sec:org695d3d1}
这个题目有个特殊的条件:任意 k 个集合的并的元素不少于 k 。也就是说,无论是哪些集合拿出来,元素都不少于集合的个数。于是,我们可以构造一个元素到集合的双射,跑一个完美匹配的算法即可。然后,根据这个,我们就知道,如果我们要选某一个集合,就一定要把这个集合的元素对应的集合都选取进来,否则不可能达到要求。将这个要求进行图论建模,刚好对应与最小闭合子图的问题,而最小闭合子图类似于最大闭合子图,可以构造网络流用网络流的方法解决。

\section{程序编译环境}
\label{sec:org048f3c3}
\begin{enumerate}
\item 操作系统: macOS
\item 编译器: LLVM/Clang 6.0.0
\end{enumerate}

\section{实现步骤}
\label{sec:orgf871837}
\subsection{数据读入}
\label{sec:org5a9c357}
\begin{minted}[linenos,numbersep=5pt,breaklines]{c++}
scanf("%d", &n);
for (int i = 1; i <= n; i++) {
  scanf("%d", &num_edges[i]);
  for (int j = 0; j < num_edges[i]; j++) {
    scanf("%d", &edges[i][j]);
  }
 }
for (int i = 1; i <= n; i++) {
  scanf("%d", &weight[i]);
  // Find minimum instead of maximum
  weight[i] = -weight[i];
 }
\end{minted}

首先读入每个集合的元素,和权值。因为我们后面把最小权闭合子图转化为最大权闭合子图来做,于是我们需要把权值取相反数。

\subsection{完美匹配}
\label{sec:orga3217d5}

\begin{minted}[linenos,numbersep=5pt,breaklines]{c++}
// Perfect Matching
for (int i = 1; i <= n; i++) {
  memset(visit, 0, sizeof(visit));
  match(i);
 }
\end{minted}

主函数中对每个元素调用 \texttt{matching} 函数,不断查找增广路。

\begin{minted}[linenos,numbersep=5pt,breaklines]{c++}
bool match(int cur) {
  for (int i = 0; i < num_edges[cur]; i++) {
    int another = edges[cur][i];
    if (visit[another] == 0) {
      visit[another] = 1;
      if (matching[another] == 0 || match(matching[another])) {
        matching[another] = cur;
        return true;
      }
    }
  }

  return false;
}
\end{minted}

这里通过一个 DFS 找增广路并且更新当前匹配。 \texttt{matching[num]} 代表数字 num 对应的是哪个集合。

\subsection{图论建模}
\label{sec:orge134f94}
接下来,我们构造最大闭合子图对应的网络流:正权点从源点连入,负权边向汇点连出,把前面提到的依赖关系通过一条边把点连起来。

\begin{minted}[linenos,numbersep=5pt,breaklines]{c++}
int positive_weight = 0;
for (int i = 1; i <= n; i++) {
  for (int j = 0; j < num_edges[i]; j++) {
    // if i is covered, then
    // all numbers in i should be covered,
    // link those corresponding sets
    // if (matching[edges[i][j]] != i)
    add_edge(i, matching[edges[i][j]], INF);
  }
 }
for (int i = 1; i <= n; i++) {
  if (weight[i] < 0) {
    // link to sink
    add_edge(i, n + 1, -weight[i]);
  } else {
    add_edge(0, i, weight[i]);
    positive_weight += weight[i];
  }
 }
\end{minted}

这里的 \texttt{add\_edge} 采用了网络流的 \texttt{residue} 表示方法和下标实现边的链表的方法:

\begin{minted}[linenos,numbersep=5pt,breaklines]{c++}
void add_edge(int from, int to, int cap) {
  edges_flow[++top] = edge{to, top_edges_flow[from], cap};
  top_edges_flow[from] = top;
  edges_flow[++top] = edge{from, top_edges_flow[to], 0};
  top_edges_flow[to] = top;
}
\end{minted}

其中 \texttt{top} 表示当前的边数, \texttt{top\_edges\_flow} 表示该结点最后一条边的下标,正向边的余量就是 \texttt{cap} ,反向边的余量就是 \texttt{0} 。正向边和反向边可以通过改变最低位完成。

\subsection{网络流 Dinic 算法}
\label{sec:org6db176e}
最后,在建立的图上跑 Dinic 算法。首先是对图进行 \texttt{bfs} :

\begin{minted}[linenos,numbersep=5pt,breaklines]{c++}
bool bfs() {
  for (int i = 0; i <= n + 1; i++) {
    depth[i] = -1;
  }
  std::queue<int> que;
  que.push(0);
  depth[0] = 0;
  while (!que.empty()) {
    int current = que.front();
    que.pop();
    for (int i = top_edges_flow[current]; i != 0; i = edges_flow[i].next_edge) {
      int next = edges_flow[i].to;
      if (edges_flow[i].residue > 0 && depth[next] < 0) {
        depth[next] = depth[current] + 1;
        que.push(next);
      }
    }
  }
  return depth[n + 1] > 0;
}
\end{minted}

同时检测汇点不可达的情况。然后根据得到的 \texttt{depth} 数组进行增广路的寻找:

\begin{minted}[linenos,numbersep=5pt,breaklines]{c++}
int dfs(int current, int to, int current_flow) {
  if (current == to || current_flow == 0) {
    return current_flow;
  }

  int flow = 0;
  for (int i = top_edges_flow[current]; i != 0; i = edges_flow[i].next_edge) {
    int next = edges_flow[i].to;
    if (edges_flow[i].residue > 0 && depth[next] == depth[current] + 1) {
      int result =
        dfs(next, to, min(edges_flow[i].residue, current_flow - flow));
      if (result) {
        flow += result;
        edges_flow[i].residue -= result;
        edges_flow[i ^ 1].residue += result;
        if (flow == current_flow) {
          return flow;
        }
      }
    }
  }
  if (flow == 0) {
    depth[current] = -1;
  }
  return flow;
}
\end{minted}

最后,在 \texttt{main} 中多次循环,并且最后输出最大闭合子图的结果:

\begin{minted}[linenos,numbersep=5pt,breaklines]{c++}
int max_flow = 0;
while (bfs()) {
  max_flow += dfs(0, n + 1, INF);
 }
printf("%d\n", max_flow - positive_weight);
\end{minted}

\subsection{完整代码}
\label{sec:orgad8a381}
\begin{minted}[linenos,numbersep=5pt,breaklines]{c++}
#include <memory.h>
#include <queue>
#include <stdio.h>
#include <string.h>

const static int INF = 1 << 30;

int n;
// 1~n: set
int num_edges[700] = {0};
int edges[700][700] = {{0}};
int matching[700] = {0};
int visit[700] = {0};

// 0: source
// n+1: sink
struct edge {
  int to;
  int next_edge;
  int residue;
} edges_flow[500 * 500 * 2];
int top = 1;
int top_edges_flow[700] = {0};

int weight[500] = {0};
int depth[500] = {0};
int map_set[500] = {0};

inline int min(int a, int b) { return a > b ? b : a; }

void add_edge(int from, int to, int cap) {
  edges_flow[++top] = edge{to, top_edges_flow[from], cap};
  top_edges_flow[from] = top;
  edges_flow[++top] = edge{from, top_edges_flow[to], 0};
  top_edges_flow[to] = top;
}

bool match(int cur) {
  for (int i = 0; i < num_edges[cur]; i++) {
    int another = edges[cur][i];
    if (visit[another] == 0) {
      visit[another] = 1;
      if (matching[another] == 0 || match(matching[another])) {
        matching[another] = cur;
        return true;
      }
    }
  }

  return false;
}

bool bfs() {
  for (int i = 0; i <= n + 1; i++) {
    depth[i] = -1;
  }
  std::queue<int> que;
  que.push(0);
  depth[0] = 0;
  while (!que.empty()) {
    int current = que.front();
    que.pop();
    for (int i = top_edges_flow[current]; i != 0; i = edges_flow[i].next_edge) {
      int next = edges_flow[i].to;
      if (edges_flow[i].residue > 0 && depth[next] < 0) {
        depth[next] = depth[current] + 1;
        que.push(next);
      }
    }
  }
  return depth[n + 1] > 0;
}

int dfs(int current, int to, int current_flow) {
  if (current == to || current_flow == 0) {
    return current_flow;
  }

  int flow = 0;
  for (int i = top_edges_flow[current]; i != 0; i = edges_flow[i].next_edge) {
    int next = edges_flow[i].to;
    if (edges_flow[i].residue > 0 && depth[next] == depth[current] + 1) {
      int result =
        dfs(next, to, min(edges_flow[i].residue, current_flow - flow));
      if (result) {
        flow += result;
        edges_flow[i].residue -= result;
        edges_flow[i ^ 1].residue += result;
        if (flow == current_flow) {
          return flow;
        }
      }
    }
  }
  if (flow == 0) {
    depth[current] = -1;
  }
  return flow;
}

int main() {
  scanf("%d", &n);
  for (int i = 1; i <= n; i++) {
    scanf("%d", &num_edges[i]);
    for (int j = 0; j < num_edges[i]; j++) {
      scanf("%d", &edges[i][j]);
    }
  }
  for (int i = 1; i <= n; i++) {
    scanf("%d", &weight[i]);
    // Find minimum instead of maximum
    weight[i] = -weight[i];
  }
  // Perfect Matching
  for (int i = 1; i <= n; i++) {
    memset(visit, 0, sizeof(visit));
    match(i);
  }
  // Maximum flow
  int positive_weight = 0;
  for (int i = 1; i <= n; i++) {
    for (int j = 0; j < num_edges[i]; j++) {
      // if i is covered, then
      // all numbers in i should be covered,
      // link those corresponding sets
      // if (matching[edges[i][j]] != i)
      add_edge(i, matching[edges[i][j]], INF);
    }
  }
  for (int i = 1; i <= n; i++) {
    if (weight[i] < 0) {
      // link to sink
      add_edge(i, n + 1, -weight[i]);
    } else {
      add_edge(0, i, weight[i]);
      positive_weight += weight[i];
    }
  }
  int max_flow = 0;
  while (bfs()) {
    max_flow += dfs(0, n + 1, INF);
  }
  printf("%d\n", max_flow - positive_weight);
  return 0;
}

\end{minted}

\section{遇到的问题和得到的收获}
\label{sec:orge7df106}
遇到的问题是,首先在编写完美匹配的代码的时候,记错了这个算法的一个细节,导致调试了很久。第二个就是,如果要通过异或来得到反向边的下标,需要注意加入边时下标是否满足这个。写代码就是这样,总会在意想不到的地方出现自己的失误。
\end{document}
